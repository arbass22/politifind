\documentclass{article}
\usepackage[utf8]{inputenc}
\usepackage[margin=1in]{geometry}
\usepackage{graphicx}
\graphicspath{{final-images/}}
\usepackage{hyperref}
\hypersetup{
    colorlinks=true,
    linkcolor=blue,
    filecolor=magenta,
    urlcolor=black,
}
\newcommand{\n}{\noindent}

\begin{document}
\title{Sudo rm -rf / \\ \large{Politifind}}
\author{Fall 2017}
\date{}
\maketitle

\n\textbf{Overview}: \textbf{NOTE}: With the implementation of our team choice searching functionality, our server depends on the \verb|algoliasearch_django| module that can be installed by running \verb|$ pip install algoliasearch-django| \\

Politifind aims to make data about all the happening in Congress accessible and intuitive to the average user. We have designed a web app to nicely compile all such data into pages that can easily be navigated too and connected with each other. Most government websites have this data in difficult to read formats, or in formats that are not intuitive/connected. Politifind aims eliminates this problem, and allows people to stay informed about all that's going on. \\

\n\textbf{Members}: Andrew Bass, Matthew Bissaillon, Matthew Gramigna, Justin Kennedy \\

\n\textbf{Github}: \url{https://github.com/arbass22/politifind} \\

\n\textbf{User Interface}: \\\\
\textbf{Home page:} This page shows the recent events on bills as well as votes.  If the user is logged in then it can show more customized
information such as updates on bills the user is subscribed to.  The search bar at the top allows searching of bills, committees, and politicians at the same time. \\

\includegraphics[width=6in]{homepage}\\\\
\textbf{Politician Page:} This page shows details on a selected politician, such as recent votes, sponsored bills, and social media accounts.   There are multiple tabs that can be used to browse for more information.   Everything like bills and committees is linked to the right pages for easy browsing. In addition, users can subscribe to the politician.\\

\includegraphics[width=6in]{politician}\\\\
\textbf{Bill Page:} This page details on each bill.  It allows users to see the summary, sponsors, actions, and similar bills.  It also shows congressional vote data and allows Politifind users to vote on a bill, adding an optional comment to their representatives in congress. Users can also subscribe to the bill by clicking the bell icon. \\

\includegraphics[width=6in]{bill}\\\\
\textbf{Committee Page:} This page shows members of a committee, bills under consideration, and subcommittees.  Users can subscribe to committees as well. \\

\includegraphics[width=6in]{committee}\\\\
\textbf{Profile Page:} This page allows users to update basic details of their profile, such as email, picture, party.  It also shows the user their subscriptions. \\

\includegraphics[width=6in]{profile}\\\\
\textbf{Search Page:} If the typeahead in the search bar does not yield the result the user is looking for, they can hit enter and be taken to this search page.  It will show all of the results (bills, politicians, and committees), rather than just the top ones. \\

\includegraphics[width=6in]{search}\\





\n\textbf{Data Model}: The following diagram is our updated data model for Politifind with the following descriptions: \\ 

\n\textit{Profile}: Includes authentication fields for the django User as well as information about this user's politifind profile. \\
\n\textit{Politician}: Represents a member of congress in either the house or senate. \\
\n\textit{Bill}: A congressional bill.\\
\n\textit{PoliticianVote}: Indicates how a specific Politician voted on a specific bill. \\
\n\textit{UserVote}: Indicates how a specific politifind user voted on a specific bill.\\
\n\textit{Committee}: Represents a committee in congress. \\
\n\textit{SubCommittee}: Represents a subcommittee of an existing politifind Committee object. \\
\n\textit{CommitteeMembership}: Indicates that a specific Politician is a member of a specific Committee. \\
\n\textit{BillCommittee}: Indicates that a specific Committee introduced a specific Bill. \\
\n\textit{BillSponsorship}: Indicates which Politician sponsored a specific Bill. \\
\n\textit{BillAction}: An action that has happened on a specific Bill. \\
\n\textit{UserPoliticianSubscription}: Indicates that a politifind user subscribed to a specific Politician. \\
\n\textit{UserBillSubscription}: Indicates that a politifind user subscribed to a specific Bill.\\
\n\textit{UserCommitteeSubscription}: Indicates that a politifind user subscribed to a specific Committee.\\

\includegraphics[scale=0.5]{politifind-data-model-UPDATED.png} \\

\n\textbf{URL Routes/Mappings}: 
\begin{center}
\begin{tabular}{ l | l | l }
route & name & description \\
\hline
r`\^{}\$' & index & route users to the home page \\
r`\^{}bill/(.*)/\$' & bill & individual bill page\\
r`\^{}politician/(.*)/(bills$\mid$votes)*' & politician & individual politician page\\
r`\^{}politicians/\$' & politicians & list of all politicians\\
r`\^{}bills/\$' & bills & list of all bills\\
r`\^{}committee/(.*)/(bills$\mid$subcomittees)*\$' & committee & individual committee page\\
r`\^{}committees/\$' & committees & list of all committees\\
r`\^{}profile/\$' & profile & user's profile page\\
r`\^{}search/\$' & search & display's search results\\
r`\^{}accounts/' & accounts & adds login/logout/etc urls\\
r`\^{}vote/(.*)/(yay$\mid$nay)*/' & vote & view individual vote page\\
r`\^{}subscribe/\$' & subscribe & user can subscribe\\
r`\^{}unsubscribe/\$' & unsubscribe & user can unsubscribe\\
r`\^{}profile/update/' & update profile & have a user update their profile page\\
\end{tabular}
\end{center}

\n\textbf{Authentication/Authorization}: We added a one-to-one mapping between Django's User model and our custom Profile model, allowing a user to specify additional information, such as political party, profile picture, etc.. Our users can browse Politifind without logging in, but when they do they have some additional functionality, such as subscribing to various politicians, bills, and committees, or voting on bills to see how they stack up against all politfind users and/or congress. We also enable to profile view to a logged in user where they can view all of their subscriptions and change their profile info. Lastly, the homepage is templated differently based on whether or not a user is logged in. We have one permissions group for a user that allows them to modify their info, or add votes and subscriptions, but it does not allow them to for example change the data of a bill, as it shouldn't. \\

\n\textbf{Team Choice}: For our team choice, we decided to integrate the Algolia Search API into our app to allow the user to very quickly search through all the data in politifind. To do this, we added two features. The first is an autocomplete search that reacts to what the user types in the search bar and displays results in a dropdown. this searching can happen from any of our views. The other feature is an actual search page. If a user does not want to look at the dropdown results, they can hit enter and it will bring them to a search page displaying the results from their query. We used the Algolia Javascript client for the autocompleting, and the algoliasearch-django pip module mentioned before for the search page. \\

\n\textbf{Conclusion}: Overall, we worked well as a team, understanding what needed to get done and the various deadlines. We are happy with our final project, and would like to continue working on it in the future. One of the main takeaways we got from this project was the importance of clearly defining efficient and sensible data models. Having an understanding of the exact type of data you want to template makes templating it even easier. One difficulty we had was leaving our background on component-based Javascript frameworks (Vue and React) and moving towards the Django templating. It was initially difficult to make this transition, but after project 2 we were pretty solid with this. As mentioned in a previous writeup, we also had a large technical hurdle gathering all of the data required for our project. After we got past this hurdle, the project really started to come together though. Going into the project, we wished we spent more times thoroughly reading the homeworks when we were doing them, as it would've made certain steps of the projects much easier and would have saved lots of time. \\

\end{document}
