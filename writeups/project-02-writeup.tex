\documentclass{article}
\usepackage[utf8]{inputenc}
\usepackage[margin=1in]{geometry}
\usepackage{hyperref}
\hypersetup{
    colorlinks=true,
    linkcolor=blue,
    filecolor=magenta,
    urlcolor=black,
}
\newcommand{\n}{\noindent}

\begin{document}
\begin{center}
\subsection*{Sudo rm -rf / Project 2 Writeup}
\end{center}

\subsubsection*{Team Writeup}

\textbf{Overview}

The politifind project aims to make political data more accessible to the average person. The project aims to be easy searchable by items such as politicians, bills, and committees. Other data such as how politicians vote on bills and actions on bills will also be available. In addition, users will eventually be able to provide their own input on certain items, such as the popularity of a bill, and this will be available to all viewers.

\n\textbf{Members}: Andrew Bass, Matthew Bissaillon, Matthew Gramigna, Justin Kennedy

\n\textbf{Github}: \url{https://github.com/arbass22/politifind}

\n\textbf{Design Overview}

Our models were defined according for the most part to our data modeling diagram. Some fields had to be slightly altered, and some new ones added based on the data that was available in the API. There were also certain aspects of the data that we didn't feel were crucial to this specific project submission.

Many of our pages have multiple tabs, for example, for Politicians we have tabs for recents votes and sponsored bills.  We built base templates for the pages and then extended for each tab.  We also have been working to build up a collection of smaller templates that we use as reusable components that can be used in different parts of the site to maintain a consistent look and reduce code repetition.

\n\textbf{Problems/Successes}

Some of the problems we experienced was having to change our models a couple of times based on the available data from the ProPublica API. Then, since we all individually wrote up a sample page for the mock UI, we spent a lot of time making the templates uniform for UI style and necessary content before we could then add the templatable features. Overall, managing the large amount of data for the project was difficult, but we nicely handled this. We also had two people work on the templates and two people work on the back end data collection, so teamwork was definitely solid for us, because no one was left completely alone doing an item of work. That being said, we definitely could have communicated better in terms of what data was in the database versus what's in the template, at least in terms of variable names and such. We found that as we started to hook up our templates to the mock data in the database, a lot of the variables in the templates needed to be changed due to the lack of consistency between the two parts, a problem which could have likely been avoided with better communication.

\pagebreak

\subsubsection*{Andrew Bass Individual Writeup}
I worked on implementing the routing and views for the site.  I mainly worked on the Politician and Committee pages.  In order to support the dynamic data as well are to create reusable components, I created smaller template pieces that are in the "templates/componenets" directory.  These will be able to used in multiple places on the site for our more standard UI components.  This will help us prevent duplicate code as well as achieve a more uniform look.  These pages have multiple subsections so I created base templates for them, that are extended in the subpages. Evetually, we may want to do the section switched using javascript instead of reloading the rest of the page. I also assisted in hooking up the data into these views, adapting to the types of data that are currently available to us (with more to come in the future).\\

\n\textbf{Contribution: 25\%}

\pagebreak

\subsubsection*{Matthew Bissaillon Individual Writeup}
I worked on gathering the data for the app along with Matthew Graminga, which involved me writing some scripts to save mock data into the database. This also involved me adding some data directly through the admin page. I also edited some of the views to make sure the templates were showing the correct data. I also started up the project and performed the first initial steps to work the politifind app. This includes setting up the admin page. \\

\n\textbf{Contribution: 20\%}
\pagebreak

\subsubsection*{Matthew Gramigna Individual Writeup}

First, I defined all of the models for the app in \verb|models.py|. I also worked (along with Matthew Bissaillon) on gathering all of the data used for these models. Most of this data, such as bills politicians or committees, was generating using API calls to the pro-publica congress API that we planned on using for the app.  Another portion of the data, mainly the foreign key-heavy models, were generated randomly e.g. A random politician was chosen to be the sponsor of a given bill. I helped write the scripts that both made the API calls and saved the data in the database, along with the scripts that randomly generate relationships between certain types of data.

I also helped connect the data in the database with the templates that are rendered. Initially, junk data was used in the python scripts to render a specific view. I changed that junk data to be calls to the database, and helped refactor the templates to ensure that the correct variable names and such are being used to render the mock data into the templates. \\

\n\textbf{Contribution: 30\%}

\pagebreak

\subsubsection*{Justin Kennedy Individual Writeup}

First, I restructured the \verb|views.py| file to a python module in the \verb|views| directory so that each view could have its own file.  Then, I wrote some of the urls in \verb|urls.py| (index, search, bills, committees, politicians, bill, and profile).  I also wrote the starter view file for the previously listed urls.  I was responsible for writing, styling, and templating for \verb|base_template.html| (the navbar) \verb|profile.html| and \verb|bill.html|, and \verb|search.html|.  I also partially contributed to \verb|index.html|, \verb|bills.html|, \verb|politicians.html|, and \verb|committees.html|.\\

\n\textbf{Contribution: 25\%}

\end{document}
